\documentclass[11pt, letterpaper]{article}

\usepackage[usenames,dvipsnames]{color} % Required for custom colors

\usepackage{
  amsmath,
  amssymb,
  amsthm,
  caption,
  enumerate,
  float,
  graphicx,
  setspace
}

\doublespacing
\setlength\parindent{0pt}

\setlength{\textwidth}{7in}
\setlength{\topmargin}{-0.8in}
\setlength{\textheight}{10in}
\setlength{\oddsidemargin}{-0.5in}
\setlength{\evensidemargin}{-0.5in}

\begin{document}

\noindent
CSI 4130 \hfill Winter 2017 \\
Computer Science \hfill University of Ottawa \\
Isaac Gollish \hfill Student Number: 6847497\\
\begin{center}
    \textbf{Assignment 4} \\
\end{center}

\hrule

\section*{Purpose}
The goal of this assignment was to make use of a number of methods learned in
class, namely animation, lighting and texturing. Unfortunately, there was not
enough time to finish the texturing, but it is hoped that this can be easily
done.


\section*{Brief description}

The object seen in the screenshot is a Möbius strip, a non-orientable 2D
surface. When animated, the green sphere rotates around the figure. There is a
diffuse white light in the upper right corner of the viewing volume. The surface
was created by scratch, as well as the calculation of the normals, the
calculation of the lighting, and the path of the sphere (although the sphere is
using a one-line call to a GLUT method).

There is a screenshot in \texttt{description/screenshot.png}.

\section*{Issues}

The zebra texture of the Möbius strip is actually unintentional and is due to
some problems with calculating the normals. Commenting out the code for the
normals buffer seems to restore the proper lighting, although it is not clear why
exactly at the moment.

Due to time constraints, texturing is not complete. If texturing the strip does
not take long, it will be done.

\end{document}
